The file {\bfseries{\mbox{\hyperlink{cmsis__os_8h}{cmsis\+\_\+os.\+h}}}} is a template header file for a C\+M\+S\+I\+S-\/\+R\+T\+OS compliant Real-\/\+Time Operating System (R\+T\+OS). Each R\+T\+OS that is compliant with C\+M\+S\+I\+S-\/\+R\+T\+OS shall provide a specific {\bfseries{\mbox{\hyperlink{cmsis__os_8h}{cmsis\+\_\+os.\+h}}}} header file that represents its implementation.

The file \mbox{\hyperlink{cmsis__os_8h}{cmsis\+\_\+os.\+h}} contains\+:
\begin{DoxyItemize}
\item C\+M\+S\+I\+S-\/\+R\+T\+OS A\+PI function definitions
\item struct definitions for parameters and return types
\item status and priority values used by C\+M\+S\+I\+S-\/\+R\+T\+OS A\+PI functions
\item macros for defining threads and other kernel objects
\end{DoxyItemize}

{\bfseries{Name conventions and header file modifications}}

All definitions are prefixed with {\bfseries{os}} to give an unique name space for C\+M\+S\+I\+S-\/\+R\+T\+OS functions. Definitions that are prefixed {\bfseries{os\+\_\+}} are not used in the application code but local to this header file. All definitions and functions that belong to a module are grouped and have a common prefix, i.\+e. {\bfseries{os\+Thread}}.

Definitions that are marked with {\bfseries{C\+AN BE C\+H\+A\+N\+G\+ED}} can be adapted towards the needs of the actual C\+M\+S\+I\+S-\/\+R\+T\+OS implementation. These definitions can be specific to the underlying R\+T\+OS kernel.

Definitions that are marked with {\bfseries{M\+U\+ST R\+E\+M\+A\+IN U\+N\+C\+H\+A\+N\+G\+ED}} cannot be altered. Otherwise the C\+M\+S\+I\+S-\/\+R\+T\+OS implementation is no longer compliant to the standard. Note that some functions are optional and need not to be provided by every C\+M\+S\+I\+S-\/\+R\+T\+OS implementation.

{\bfseries{Function calls from interrupt service routines}}

The following C\+M\+S\+I\+S-\/\+R\+T\+OS functions can be called from threads and interrupt service routines (I\+SR)\+:
\begin{DoxyItemize}
\item \mbox{\hyperlink{cmsis__os_8h_a3de2730654589d6c3559c4b9e2825553}{os\+Signal\+Set}}
\item \mbox{\hyperlink{cmsis__os_8h_ab108914997c49e14d8ff1ae0d1988ca0}{os\+Semaphore\+Release}}
\item \mbox{\hyperlink{cmsis__os_8h_ab4bc93bf17f94ca99363e53e1a763fb5}{os\+Pool\+Alloc}}, \mbox{\hyperlink{cmsis__os_8h_ab22c2f194e0f0b05494aa0e9c97d7a51}{os\+Pool\+C\+Alloc}}, \mbox{\hyperlink{cmsis__os_8h_a4a861e9c469c9d0daf5721bf174f8e54}{os\+Pool\+Free}}
\item \mbox{\hyperlink{cmsis__os_8h_ac0dcf462fc92de8ffaba6cc004514a6d}{os\+Message\+Put}}, \mbox{\hyperlink{cmsis__os_8h_a6c6892b8f2296cca6becd57ca2d7e1ae}{os\+Message\+Get}}
\item \mbox{\hyperlink{cmsis__os_8h_ac985d7f260b80d7891157697b760aa02}{os\+Mail\+Alloc}}, \mbox{\hyperlink{cmsis__os_8h_a9e492e5839bf0d2bf66492e70eae3ddb}{os\+Mail\+C\+Alloc}}, \mbox{\hyperlink{cmsis__os_8h_ac6ad7e6e7d6c4a80e60da22c57a42ccd}{os\+Mail\+Get}}, \mbox{\hyperlink{cmsis__os_8h_a485ef6f81854ebda8ffbce4832181e02}{os\+Mail\+Put}}, \mbox{\hyperlink{cmsis__os_8h_a27c1060cf21393f96b4fd1ed1c0167cc}{os\+Mail\+Free}}
\end{DoxyItemize}

Functions that cannot be called from an I\+SR are verifying the interrupt status and return in case that they are called from an I\+SR context the status code {\bfseries{os\+Error\+I\+SR}}. In some implementations this condition might be caught using the H\+A\+RD F\+A\+U\+LT vector.

Some C\+M\+S\+I\+S-\/\+R\+T\+OS implementations support C\+M\+S\+I\+S-\/\+R\+T\+OS function calls from multiple I\+SR at the same time. If this is impossible, the C\+M\+S\+I\+S-\/\+R\+T\+OS rejects calls by nested I\+SR functions with the status code {\bfseries{os\+Error\+I\+S\+R\+Recursive}}.

{\bfseries{Define and reference object definitions}}

With {\bfseries{\#define os\+Objects\+External}} objects are defined as external symbols. This allows to create a consistent header file that is used throughout a project as shown below\+:

{\itshape Header File} 
\begin{DoxyCode}{0}
\DoxyCodeLine{\textcolor{preprocessor}{\#include <\mbox{\hyperlink{cmsis__os_8h}{cmsis\_os.h}}>}                                         \textcolor{comment}{// CMSIS RTOS header file}}
\DoxyCodeLine{}
\DoxyCodeLine{\textcolor{comment}{// Thread definition}}
\DoxyCodeLine{\textcolor{keyword}{extern} \textcolor{keywordtype}{void} thread\_sample (\textcolor{keywordtype}{void} \textcolor{keyword}{const} *argument);             \textcolor{comment}{// function prototype}}
\DoxyCodeLine{\mbox{\hyperlink{cmsis__os_8h_a1537e80813785bebbc3cbab3226eb04f}{osThreadDef}} (thread\_sample, \mbox{\hyperlink{cmsis__os_8h_a7f2b42f1983b9107775ec2a1c69a849aa193b650117c209b4a203954542bcc3e6}{osPriorityBelowNormal}}, 1, 100);}
\DoxyCodeLine{}
\DoxyCodeLine{\textcolor{comment}{// Pool definition}}
\DoxyCodeLine{\mbox{\hyperlink{cmsis__os_8h_a87b471d4fe2d5dbd0040708edd52771b}{osPoolDef}}(MyPool, 10, \textcolor{keywordtype}{long});}
\end{DoxyCode}


This header file defines all objects when included in a C/\+C++ source file. When {\bfseries{\#define os\+Objects\+External}} is present before the header file, the objects are defined as external symbols. A single consistent header file can therefore be used throughout the whole project.

{\itshape Example} 
\begin{DoxyCode}{0}
\DoxyCodeLine{\textcolor{preprocessor}{\#include "osObjects.h"}     \textcolor{comment}{// Definition of the CMSIS-\/RTOS objects}}
\end{DoxyCode}



\begin{DoxyCode}{0}
\DoxyCodeLine{\textcolor{preprocessor}{\#define osObjectExternal   // Objects will be defined as external symbols}}
\DoxyCodeLine{\textcolor{preprocessor}{\#include "osObjects.h"     // Reference to the CMSIS-\/RTOS objects}}
\end{DoxyCode}
 